
The 2013 Nuclear Engineering Student Delegation was very successful in educating
policymakers on the needs for supporting nuclear engineering education and the
members of the delegation on the political process. During the delegations time
in Washington, D.C., members were able to speak for nuclear science and
engineering students throughout the country and show their enthusiasm for
nuclear professions. This years delegation was fortunate to meet with some very
influential members and advocates of the nuclear community. The NESD was also
very successful on Capitol Hill as they were able to visit or meet with the
offices of numerous members of the House of Representatives and most sitting
members of the Senate during their time in Washington, D.C.

The primary goal of this year’s delegation was to express to policymakers the
need for the Integrated University Program (IUP) that provides fellowships and
scholarships through DOE/NE and NRC to nuclear science and engineering
students. Currently, the IUP has been funded under the House's Water \& Energy
Appropriations bill. It is still uncertain whether or not the senate will
include the IUP in their appropriations bill.p

All members of the 2013 NESD were very pleased with their experience in
Washington, D.C. Delegates who will be in school next year look forward to the
possibility of being part of the 2014 NESD, which will be chaired by Nicholas
Thompson (Rensselar Polytechnic Institute). Finally, the members of the
delegation would like to thank the Nuclear Energy Institute, American Nuclear
Society, and all of the universities of the delegates for their help and
support.
