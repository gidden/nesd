
2013 was another successful year for the Nuclear Engineering Student
Delegation. The group met with policy developers from the nuclear industry,
policy implementers at the DOE, NRC, and State Department, and with policy
makers in both houses of congress as well as OMB. Visiting the State Department
was a new venture for the delegation this year and was very well received by
both the delegates and our hosts. The OMB meeting this year included, for the
first time, a OMB's NSF funding lead, allowing discussion to focus on the types
of funding available for science students and students of nuclear science and
technology more specifically. The delegation was also able to interface over
dinner with two congressional aides on the Senate Committee on Energy and
Natural Resources working specifically on the Nuclear Waste Administration Act
as well as two aides on the House's Energy Committee, including the current ANS
Fellow. Of special note, the delegation was able to meet with four out of the
five commissioners of the NRC.

NESD's policy statement emphasized two subjects, continued funding of the
Integrated University Program (IUP), which provides fellowships and scholarships
through DOE/NE and NRC to nuclear science and engineering students, and passage
of the Nuclear Waste Administration Act (NWAA). A number of other issues were
addressed, including domestic fusion research funding, energy policy, nuclear
export agreements, and neutron detectors for port security. The full policy
statement is provided below in Appendix \ref{sec:statement}.

NESD would like to thank our sponsers: NEI, ANS, and the universities of each
delegate. Without your continued support, this venture would not be possible. As
always, we look forward to next year's delegation, which will be chaired by
Nicholas Thompson (Rensselar Polytechnic Institute).
