
\subsection*{Sunday, July 7th}

\subsubsection*{Policy Statement Writing}

The entire group gathered in a meeting room at the hotel and spent the first
hour or so talking about what they considered to be the important issues facing
nuclear engineering education. After everyone had a chance to express their
thoughts, the Chairs gave a quick recap of everything that had been
discussed. The delegates emphasized two subjects, continued funding of the
Integrated University Program (IUP) and passage of the Nuclear Waste
Administration Act (NWAA). A number of other issues were addressed, including
domestic fusion research funding, energy policy, nuclear export agreements, and
neutron detectors for port security. After the break for lunch, the delegates
divided back into groups and drafted each section. At the end of the day the
group edited the sections together. The delegates read the combined document
agreed upon its content by consensus. The full policy statement is provided in
Appendix \ref{sec:statement}.

\subsection*{Monday, July 8th}

\subsubsection*{AREVA Meeting} 

On Monday morning, the delegates had their first meeting with the governmental
affairs staff of Areva. It began with a general presentation by the Vice
President of Governmental Affairs on Areva's international business
portfolio. The Areva staff gave a presentation about the MOX project and the
importance to our joint arms reduction commitments with Russia. They also
discussed the business importance of being active in policy discussions in
D.C. The morning concluded with a discussion with Mary Alice Hayward on her
experiences in nonproliferation and the importance of technical expertise in
international agreements.

\subsubsection*{NEI Meeting} 

On Monday afternoon, the delegation visited the Nuclear Energy Institute. Leslie
Barbour and other staff members described the function of NEI and how the
organization operates. They then discussed the importance of building and
maintaining relationships with congressional staff. They also explained NEI's
surveys and data on the nuclear workforce. We ended the meeting discussing our
policy statement and received helpful feedback.

\subsubsection*{Dinner with Congressional Fellows}

On Monday evening, the delegation had dinner with Lara Pierpoint (AAAS
Congressional Fellow working for Senator Ron Wyden, D-OR) and Ron Faibish
(Science Fellow for the Senate Committee on Energy and Natural Resources). These
young scholars shared their experiences on the Hill and specifically spoke about
their efforts in crafting the Nuclear Waste Administration Act.

\subsection*{Tuesday, July 9th}

\subsubsection*{DOE Meeting}

On Tuesday morning, the delegation visited the Department of Energy to meet with
the Office of Nuclear Energy (NE). The program leads for NE’s Fuel Cycle R\&D,
Light Water Reactor Technologies, and Nuclear Engineering University Program
(NEUP) discussed the myriad of funding and programmatic opportunities for
research provided by DOE NE. The staff discussed some of the priorities in
specific research areas and the importance of IUP funding. Brad Williams spoke
about the various forms of funding for graduate education, discussing the DOE
NEUP Scholarships and Fellowships (funded through the IUP) and the research
grants that provide graduate research assistantships at specific
universities. We then had lunch with Dr. Pete Lyons, Assistant Secretary of
Energy for Nuclear Energy, who shared his perspective on how D.C. operates and
the importance of nuclear engineering research to our nation's future.

\subsubsection*{NRC Meeting}

On Tuesday afternoon, the delegation went to the Nuclear Regulatory Commission
headquarters in Rockville, MD. The delegation met with NRC education staff, who
described the various methods by which the NRC provides educational enrichment
opportunities to universities - through graduate fellowships, undergraduate
scholarships, and curriculum development grants. The staff discussed internal
metrics used to ensure the effectiveness of the programs and reiterated that
none of these programs would be possible without the IUP. We also had the great
fortune of meeting with three of the commissioners on Tuesday afternoon
Commissioner George Apostolakis, Commissioner William C. Ostendorff, and
Chairman Allison M. Macfarlane. We had very robust discussions with all three
commissioners about many different topics, including Small Modular Reactors,
Gen-IV technology, Linear No-Threshold Dose, IUP funding, and commercial
reprocessing. The commissioners shared diverse perspectives from varied
backgrounds, but all are filtered through the lens of the NRC.

\subsection*{Wednesday, July 10th}

\subsubsection*{Department of State Meeting}

One Wednesday morning, the delegation met Dr. Gilbert Brown and Ryan Taugher at
the Department of State. Dr. Brown, a professor at University of Mass.-Lowell
and current Foster Fellow at the Department of State, explained the concept
``Team USA'' and the importance of the international framework on nuclear
security. Mr. Taugher explained the structure of the Department of State and the
framework for the Partnership for Nuclear Security (PNS). Through the PNS, the
Department of State establishes cooperative partnerships with other nations to
support the peaceful use of nuclear energy and achieve mutually beneficial
nuclear safety, security, and nonproliferation objectives.
 
\subsubsection*{OMB Meeting}

The Delegation’s annual meeting with the OMB on Wednesday afternoon was the most
notable in recent memory. Ms. Christine MacDonald, one of the employees
responsible for allocating DOE funds, discussed current budgeting. A
representative from the National Science Foundation (NSF) discussed the
President’s STEM initiative, NSF fellowships, NEUP, and IUP. The Delegation
provided recent statistical evidence showing that the majority of nuclear
engineering graduate students work for the U.S. Government after completing
their studies, effectively communicated how NEUP and IUP have helped increase
the attractiveness of the nuclear field to graduate students, and provided
numerous examples of how IUP has allowed students to conduct important nuclear
research that would not otherwise be funded.
 
\subsubsection*{Dinner with NRC Commissioner Magwood}

On Wednesday evening, the Delegation had dinner with Commissioner Magwood, who
cares deeply about student issues. We discussed our meetings with the other
three Commissioners, recent events in the nuclear industry (shutdown of San
Onofre and Kewaunee), and the IUP and the NRC’s role in education and workforce
development. Magwood imparted wisdom from his past experiences as chairman of
the Generation IV International Forum (GIF) and working with the Fast Flux Test
Facility (FFTF).

\subsection*{Thursday, July 11th and Friday, July 12th}

\subsubsection*{Hill Visits}

On Thursday and Friday, the Delegation accomplished its main objective on the
Hill. Delegates met with or dropped by the offices of all 100 Senators and 42
Representatives. The full list of House offices visited is provided in Appendix
\ref{sec:house}. The delegates also met with 5 senators (Maria Cantwell, D-WA;
Patty Murray, D-WA; Jeff Merkley, D-OR; Mike Lee, R-UT; Tom Harkin, D-IA) and 5
representatives (Ralph Hall, R-TX; Steve Stivers, R-OH; Chris Gibson, D-NY; John
Garamendi, D-CA; Bill Flores, R-TX) in person. The remainder of meetings took
place with congressional staffers and committees. Our presence was well received
and there were many conversations and an overall interest in learning more,
leaving us with the impression that our efforts indeed left a mark of influence
this year. In particular, as our visits earlier in the week to the NEI, DOE-ONE,
and NRC showed broad support for the NWAA, many of the Congressional offices
were interested in learning more about the bill or already supported it. Some
memorable discussions included one with Senator Feinstein’s office, which is
working on the NWAA. They stated that nuclear waste storage issues are more
political than technological, and they are open to implementing nuclear power in
the future to reduce carbon emissions if a waste storage solution is adopted. In
addition to discussion of our statement, delegates also had the opportunity to
discuss other nuclear-related topics and applications such as nuclear
desalination and nuclear development of U.S. oil shale. Delegates established
relationships that will be extremely valuable in the future, and we look forward
to observing the Delegation’s impact as the year progresses.

\subsubsection*{Thursday Dinner}

Thursday night we were pleased to have dinner with ANS/AAAS fellow Vincent
Esposito and Annie Caputo of the Senate Committee on Environment and Public
Works. Mr. Esposito gave some great career advice and talked about some of the
legislation he’s been working on.  

\subsubsection*{Friday Breakfast}

The following morning, we had breakfast with Leslie Barbour of NEI and Craig
Piercy, the ANS Washington Representative. Mr. Piercy talked about the ongoing
challenges with Yucca mountain and the need for people with technical experience
coming to Washington.
