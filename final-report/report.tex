% 
% Document: NESD 2013 Final Report
% Author: Matthew Gidden
% Date: Sept. 10, 2013
% Notes: Andrew Cartas provided an initial Latex draft 
%        for his UF report, which was used as a template.
% 

\documentclass[12pt]{article}

%%%%%% preamble %%%%%%%%%%
\usepackage{amssymb}
\usepackage{amsmath}
\usepackage{amsthm}
\usepackage[left=1in,top=1in,bottom=1in,right=1in]{geometry}
\usepackage[pdftex]{graphicx}
\usepackage{url}
\usepackage{float}
\usepackage{indentfirst}
\usepackage{setspace}
\usepackage{breqn}
\usepackage[loose]{subfigure}
\providecommand{\e}[1]{\ensuremath{\times 10^{#1}}}
\usepackage[labelformat=empty]{caption}


%%%%%%%%%%%%%%%%%%%%%%%%%%%%%%%%%%%%%%%%%%%%%%%%
\begin{document}

%%%%%% titlepage %%%%%%%%%%
\begin{center}
\begin{figure}[hbtp]
\centering
\includegraphics*[scale=1]{NESD_Logo.png}
\end{figure}
\begin{LARGE}
Nuclear Engineering Student Delegation\\
July 6-12, 2013\\
Final Report

\vspace{.5cm}
\end{LARGE}
\end{center}
\begin{large}
\begin{minipage}[b]{0.45\linewidth}
\centering
\begin{tabular}{r}
Matthew Gidden (Chair)\\
Mark Reed (Co-Vice Chair)\\
Nicholas Thompson (Co-Vice Chair)\\ 
Shelly Arreguin\\
Samuel Brinton\\
Lane Carasik\\
Andrew Cartas \\
Erin Dughie\\
Tom Grimes\\
Thomas Holschuh\\
Anagha Iyengar\\
Buckley ODay\\
Ekaterina Paramonova\\
Vishal Patel\\
Jeremy Pearson\\
Benjamin Reinke\\
\end{tabular}
\end{minipage}
\hspace{1cm}
\begin{minipage}[b]{0.45\linewidth}
\centering
\begin{tabular}{l}
University of Wisconsin-Madison\\
Massachusetts Institute of Technology\\
Rensselaer Polytechnic Institute\\
University of Washington, Seattle\\
Massachusetts Institute of Technology\\
University of Tennessee\\
University of Florida\\
University of New Mexico\\
Purdue University\\
Oregon State University\\
University of Tennessee\\
Massachusetts Institute of Technology\\
Massachusetts Institute of Technology\\
Texas A\&M\\
University of California, Irvine\\
Ohio State University\\
\end{tabular}
\end{minipage}
\end{large}
\thispagestyle{empty}


%%%%%% toc %%%%%%%%%%
\newpage
\tableofcontents
\thispagestyle{empty}

\begin{doublespace}

%%%%%% intro %%%%%%%%%%
\newpage
\setcounter{page}{1} 
\section{Introduction}
The 2013 Nuclear Engineering Student Delegation (NESD) was held in Washington,
DC, on July 6-12, 2013. Formed in 1994 to reinstate funding for research
reactors, the delegation continues to express the views of the student
population on nuclear science and engineering education. The delegation is
independently selected and organized with funding and support provided by the
Nuclear Energy Institute (NEI) and the American Nuclear Society (ANS). This
year’s delegation was comprised of sixteen students from twelve universities
across the country. A picture of the delegates is shown below in
Figure\ref{fig:delegates} and biographies for each delegate follow.

\begin{figure}[h]
\centering
\begin{subfigure}{.5 \textwidth}
  \centering
  \includegraphics[width=.95 \linewidth]{NESD_WH.jpg}
  \label{fig:whitehouse}
\end{subfigure}%
\begin{subfigure}{.5\textwidth}
  \centering
  \includegraphics[width=.95 \linewidth]{NESD_Ein.jpg}
  \label{fig:einstein}
\end{subfigure}
\caption{The Delegation at the White House and National Academies}
\label{fig:delegates}
\end{figure}

\subsubsection*{Matthew Gidden, University of Wisconsin - Madison (Chair)}

Matthew is a Ph.D. graduate student at the University of Wisconsin - Madison
studying nuclear engineering and energy policy. He previously attended Texas
A\&M University where he received a B.S. in nuclear engineering. Matthew
currently works in the Fuel Cycle Research Group at UW - Madison under Professor
Paul Wilson. His research interest is primarily fuel-cycle simulation and
analysis and related policy topics, such as used-fuel recycling, long-term fuel
storage, and nuclear nonproliferation.

Matthew is an active member of the American Nuclear Society, serving as the 2008
Student Conference Co-Chair as well as participating in the governance of ANS at
the national level. He has previously held internship positions at both Oak
Ridge National Laboratory working on the detection of illicit radioactive
materials and Pacific Northwest National Laboratory working on automated
verification techniques. He has also had the opportunity to work for AREVA in
Paris, France on both the transportation of used nuclear fuel as well as nuclear
reactor accident analysis.

\subsubsection*{Mark Reed, Massachusetts Institute of Technology (Co-Vice Chair)}

Mark has received his S.B. degree in Physics as well as his S.B. and
S.M. degrees in Nuclear Science and Engineering from MIT, and he is currently a
Ph.D. candidate in Nuclear Science and Engineering at MIT. His past research
includes magnetic confinement fusion and its application as a neutron source in
fission-fusion hybrid systems, enhanced fission yield modeling techniques, and
strategic plant siting in the context of seismic history. His current doctoral
research focuses on the neutronic effects of geometric distortions in fast
reactors.

He has performed reactor modeling at TerraPower and risk assessment for the
Yucca Mountain Nuclear Waste Repository at the U.S. Nuclear Regulatory
Commission. In his pre-nuclear life, he was an engineering project management
intern for the iPhone 3G at Apple and a research assistant at the Princeton
University Department of Astrophysical Sciences. Passionate about nuclear
policy, he has published a series of six articles on the history of nuclear
technology, served as a speechwriter for an elected official, and conceived the
2013 American Nuclear Society Student Conference theme ``Public Image of the
Nuclear Engineer''. In his spare time, he pursues his affinities for hiking,
making random iPhone applications, and composing awkward third-person
autobiographies.

\subsubsection*{Nicholas Thompson, Rensselaer Polytechnic Institute (Co-Vice Chair)}

Nicholas is a Ph.D. student at Rensselaer Polytechnic Institute (RPI) studying
Nuclear Engineering and Science. He graduated RPI in 2011 with a B.S. and an
M.Eng. in Nuclear Engineering. He has previously held two summer internships at
Knolls Atomic Power Laboratory, and from 2010 to 2011, was an undergraduate
researcher at the Gaerttner Linear Accelerator Center (LINAC) Laboratory at RPI.

Nick's current research focuses on using a Lead Slowing-Down Spectrometer
(LSDS)for measuring various nuclear data. In particular, two of the projects he
is working on are to make capture cross section measurements and fission
fragment distribution measurements, both with the LINAC and RPI LSDS. While
working as an undergraduate researcher, Nick helped research and perform
experiments with the RPI LSDS to assay plutonium and uranium with the goal of
nondestructively assaying spent fuel. Nick was selected as a winner of the
Innovations in Fuel Cycle Research Award and presented this research at the 2011
American Nuclear Society (ANS) Winter Conference. He was also the President of
the RPI ANS section from 2012-2013. Nick was also an NESD delegate in 2012, and
one of the Co-Vice Chairs in 2013. Some of Nick's research interests include
nuclear data, reactor design, accelerator technologies and applications, and
nuclear energy policy. Nick is an avid skier, enjoys playing billiards, and
believes that cheap, clean, reliable, safe nuclear power can help the economy
and the environment.

\subsubsection*{Shelly Arreguin, University of Washington}

Shelly is a Materials Science and Engineering (MSE) Ph.D. student at the
University of Washington in Seattle (UW). She obtained a M.S. in MSE from the UW
and bachelor's degrees in Chemistry from the University of Colorado at Boulder
(CU) and Ecology, Evolution \& Conservation Biology at UW. Her primary research
interest involves investigating the relationship of processing and properties of
materials and their performance under extreme conditions (nuclear, high
temperature, accident scenarios, etc.).

Shelly has worked extensively on developing unique processing routes from
preceramic polymers to test their capabilities in various energy applications
such as: catalysts for fuel cells (CU), hydrogen storage (National Renewable
Energy Laboratory), waste-to-energy incinerators (UW) and now high temperature
nuclear environments (UW). Previously, Shelly held an appointment at the Pacific
Northwest National Laboratory (PNNL) as a Mickey Leland Energy Fellow where she
designed chalcohalide glasses for the storage of nuclear waste streams with
increased halide content. Currently, she is at PNNL exploring the
microstructural evolution of irradiated porous and dense polymer derived SiC
ceramics for her Ph.D. thesis. When not in the lab, Shelly enjoys: cave
exploration, studying extremophiles and their associated geology, NASAs Kepler
Mission, understanding the role of bio-indicators of environmental
contamination, mushroom hunting, SCUBA and public outreach, exposing myths
vs. realities in nuclear science and technology.

\subsubsection*{Sam Brinton, Massachusetts Institute of Technology}

Samuel is completing a double M.S. program at Massachusetts Institute of
Technology in Nuclear Engineering and the Technology and Policy Program. He is a
graduate from Kansas State University with a B.S. in Mechanical and Nuclear
Engineering and a B.A. in Vocal Music Performance and a minor in Chinese
Language. His research interests are concentrated on nuclear fuel cycle system
analysis with subtopics of interest including fuel cycle economics and dry cask
storage analysis.

Samuel has had internships at the Argonne National Laboratory, Idaho National
Laboratory, and Dow Chemical Company in various projects relating to nuclear
engineering and systems analysis. He is a strong activist in a variety of civil
rights and nonproliferation issues and finds that only with a constant
interaction with our legislative representatives can we hope to make true and
lasting impacts on policy. In his spare time Samuel enjoys running, singing with
choirs and opera companies, and cheering for the K-State Wildcats and MIT
Engineers.

\subsubsection*{Lane Carasik, Texas A\&M University}

Lane is a graduate student at Texas A\&M University studying nuclear and
mechanical engineering. He recently graduated with his bachelors in Nuclear
Engineering from the University of Tennessee, Knoxville where he conducted
nuclear thermal hydraulics research under Dr. Arthur Ruggles. Lane will be a
part of the Nuclear Power Engineering Research group at TAMU under Dr. Yassin
Hassan. His research interests are in nuclear reactor thermal hydraulics and
methods development for computational fluid dynamics and heat transfer.

Lane is an active member of the American Nuclear Society and American Society of
Mechanical Engineers. Lane is currently the Vice Chair of the ANS Student
Section Committee and serving on the Executive Committee for the Thermal
Hydraulics Division. Lane has previously been the Chair of the UTK ANS Student
Section and the student chair for PHYSOR 2012. Lane has had previous internships
at Westinghouse Electric Company and Tennessee Valley Authority working on
reactor coolant systems. At Westinghouse Electric Company, Lane worked on steady
state and transient analysis for Electricite de France reactor coolant system
components and CFD method development.

\subsubsection*{Andrew Cartas, University of Florida}

Andrew is a Ph.D. student at the University of Florida studying Nuclear
Engineering where he obtained his B.S. in Nuclear Engineering in 2011. His
current research focus is on nuclear fuel fabrication, utilizing depleted
Uranium, and material performance under irradiation. Andrew has also focused his
research efforts on Silicon Carbide as a fuel additive and matrix material for
UO2.

Andrew has been heavily involved in the University of Florida ANS student
section having served two years as treasurer and is the outgoing section
president. During the summer of 2011, he was selected to be the student chair
for the 2011 ANS National Conference in Hollywood, FL. He is currently serving
on ANS National Subcommittee for Disbursement of ANS Travel Funds to
Students. Andrew has interned at Argonne National Lab and participated in the
Nondestructive Assay Applications for International Safeguards program held at
Oak Ridge National Lab.

\subsubsection*{Erin Dughie, University of New Mexico}

Erin is a PhD. graduate student at the University of New Mexico studying nuclear
engineering. She recently graduated with her undergraduate and master's degrees
from the University of Michigan in 2011 and 2012, respectively. Her research
interests include radiation detection and measurements. In the past she has
worked on detection techniques for nonproliferation, and semiconductor
devices. Her current work focuses on the detection of dark matter.

Erin is an active member of IEEE and the American Nuclear Society. She has
previously held internships at Los Alamos National Laboratory working on MCNPX
code development, and space nuclear power. In her free time, Erin is involved in
outreach activities at the local science museum. She also works with several
programs at the University of New Mexico that facilitate K-12 science and
technology activities.

\subsubsection*{Tom Grimes, Purdue University}

Tom has received a B.S. degree from Purdue University in Nuclear Engineering and
is currently a PhD graduate student at Purdue University studying Nuclear
Engineering as well as an MBA student with a focus on Entrepreneurship. Tom
currently works in the Metastable Fluid and Advanced Research Lab under
Professor Rusi Taleyarkhan. He is currently funded through the National Science
Foundation Graduate Research Fellow Program. His research interests include
nuclear non-proliferation, fluid dynamics, radiation transport, acoustics, and
materials (he holds an international patent for PLA-based coatings).

Tom's current doctoral research focuses on developing a fundamental physics
model to describe the operation of Metastable Fluid Detectors (with wider
application toward general cavitation studies e.g. making quieter submarines or
faster jet planes). His first brush with nuclear policymaking came while
evaluating Metastable Fluid Detectors for application in Radiation Portal
Monitors. Since then he has maintained a strong interest in border security and
non-proliferation policy.

\subsubsection*{Tommy Holschuh, Oregon State University}

Tommy is currently pursuing a master's degree in nuclear engineering at Oregon
State University in Corvallis, OR. He received a bachelor's in nuclear
engineering from OSU in 2013.

Tommy began performing research for Sandia National Laboratories in 2007 and
Oregon State University in 2010. His previous research areas include
magnetically-confined fusion, supercritical CO2 Brayton cycle systems,
irradiation experiments, and advanced diagnostics for the U.S. research reactor
fuel conversion program. Additionally, he is involved in OSU's student chapter
of the American Nuclear Society, in which he leads outreach programs for Boy
Scouts and other local youth groups.

For his graduate work, he will be involved in the High Temperature Test
Facility, a scaled gas reactor facility, as a research assistant under Dr. Brian
Woods. Tommy enjoys living in Oregon and likes trail running, rock climbing, and
soccer.

\subsubsection*{Anagha Iyengar, University of Tenneessee}

Anagha Iyengar is a Ph.D. student in the Nuclear Engineering department at the
University of Tennessee, Knoxville. She received her Bachelor�s of Science in
Nuclear Engineering from the University of California, Berkeley in 2012. Her
research interests lie in nuclear security, nonproliferation technologies,
international relations and energy policy. She is working on her graduate
research in collaboration with Oak Ridge National Laboratories under Dr. Jason
Hayward, and is a part of the Nuclear Materials Detection and Characterization
group. Her current research focuses on developing a passive mobile neutron
detection system for nonproliferation and security applications.

Anagha is an active member of the American Nuclear Society (ANS), Institute for
Electrical and Electronics Engineers (IEEE), and the Institute of Nuclear
Materials Management (INMM). In the past, she has had internships working on
developing and characterizing novel detection technologies at UC Berkeley,
Lawrence Berkeley National Laboratories, Lawrence Livermore National
Laboratories (Next Generation Safeguards Initiative Intern), and Sandia National
Laboratories. She is passionate about outreach efforts in local communities and
schools to advocate and encourage STEM education. She also writes for the
Nuclear Literacy Project to help dispel myths about the nuclear industry. In her
spare time, Anagha enjoys traveling, hiking, and baking.

\subsubsection*{Buck O'Day, Massachusetts Institute of Technology}

Buck is a Ph.D. Graduate student at the Massachusetts Institute of Technology
studying Nuclear Science and Engineering. He previously attended the Air Force
Institute of Technology where he earned a M.S. in Nuclear Engineering, the
University of Maryland University College where he earned a Master of
International Management, and the United States Military Academy where he earned
a B.S. in Civil Engineering. He currently studies nuclear materials detection
under MIT Senior Research Scientist Dr. Dick Lanza. His research interests
include detection of nuclear materials, nuclear policy \& security, radiation
effects, and nuclear nonproliferation.

\subsubsection*{Katia Paramonova, Massachusetts Institute of Technology}

Ekaterina (Katia) Paramonova is a Masters graduate student at the Massachusetts
Institute of Technology (MIT) studying Nuclear Science and Engineering
(NSE). She completed her Bachelors at MIT with a major in NSE and a minor in
Public Policy. Katia is working on experimental materials research on mitigating
the deposition of CRUD with Professor Michael Short. She plans on going to
France for her PhD in nuclear engineering to get a third view on the industry in
addition to her US and Russian perspectives. After completing her studies, Katia
then want to work in the industry or at a think tank for some time and then
moving on to international energy policy, with a focus on nuclear energy.

Katia is an active member of the American Nuclear Society (ANS). She is the
2013-2014 co-President of the MIT ANS section and was a co-Chair for the 2013
ANS Student Conference held at MIT. She has interned at Westinghouse, worked on
various research projects at MIT including MCNP-Serpent benchmarking work,
copper 63 and 65 capture cross section measurements, public perception of
nuclear systems modeling, and a summer project at the Harvard Managing the Atom
Center on nuclear materials in Russia. She also takes delegations of students
from MIT to Russia in the summers for conferences, assists in the Russian
SkolTech Institute nuclear center development, and is working on establishing a
student exchange program between MIT and Russia universities. Her goal is to
bring together nations as well as policy and technical experts together to help
innovation flourish.

\subsubsection*{Vishal Patel, Texas A\&M University}

Vishal is a Ph.D. graduate student at Texas A\&M University studying Nuclear
Engineering. He has received an M.S. in Nuclear Engineering from Texas A\&M and
a B.S. in Physics from The University of Texas. Vishal currently works in the
Advanced Energy Technologies group under Professor Pavel Tsvetkov. His research
interests include advanced reactor concepts and reactor control.

Vishal previously performed undergraduate research in neutron activation
analysis and was an undergraduate TA in a radiation detection lab at UT. He has
done summer work at the Center for Space Nuclear Research at the INL developing
a nuclear electric propulsion spacecraft. Outside of his academic pursuits,
Vishal enjoys weightlifting, cooking, and searching for the perfect cup of
coffee.

\subsubsection*{Jeremy Pearson, University of California - Irvine}

Jeremy is a Ph.D. graduate student at the University of California - Irvine
studying chemical engineering and used nuclear fuel recycling. He previously
attended Brigham Young University where he received a B.S. in chemical
engineering. Jeremy currently works in the Nuclear Research Group at UC Irvine
under Professor Mikael Nilsson. His research interest focuses on understanding
the sensitivity of solvent extraction processes to radiolysis in an effort to
create more robust, efficient, and economical processes which can be adopted in
a future fuel cycle that includes recycling and advanced reactor technologies.

Jeremy is an active member of the American Nuclear Society, serving on the
Education and Public Outreach committees in the San Diego ANS Local Section. In
this capacity he has given lectures on nuclear science and technology at local
high schools and worked to promote awareness of nuclear energy and technology,
especially during the NRC's evaluation for restart of the local San Onofre
Nuclear Generating Station, by organizing and hosting screenings of Switch and
Pandora's Promise at UC Irvine with their respective directors. Jeremy has also
participated with colleagues representing UCI in D.C. at the DOE's Better
Building's Case Competition presenting energy efficiency solutions to the
government's real estate portfolio managed by the GSA. In his spare time Jeremy
enjoys playing guitar, wake surfing, and playing soccer and dirt biking with his
family.

\subsubsection*{Ben Reinke, Ohio State University}

Benjamin Reinke is a Ph.D. student at the Ohio State University studying Nuclear
Engineering. He graduated from OSU with a B.S. in Physics and French and Honors
and Research Distinction in 2010. While an undergraduate, he worked in a High
Energy Density Physics laser research laboratory.

Ben is a NASA Space Technology Research Fellow. His current research focuses on
experimental and simulations for cryogenic irradiation damage
tests. Specifically Ben is establishing a cryogenic irradiation facility at the
Ohio State University Research Reactor for completing in situ damage tests on
semiconductor materials and optical fibers. Ben also works with a Material
Science professor to simulate the radiation damage in these experiments and
develop a mulit-scale model of defect annealing. Earlier in his graduate
studies, Ben worked on a Department of Energy Nuclear Engineering Program to
develop a high temperature alpha particle detector with 4H-SiC. Ben also spends
time as the president of the OSU student chapter of the American Nuclear Society
and serving as the graduate/professional student member of the OSU Board of
Trustees.


%%%%%% exec summary %%%%%%%%%%
\newpage
\section{Executive Summary}

The Nuclear Engineering Student Delegation marked another successful trip in
July 2013.  Delegates had unforgettable experiences learning about, and
participating in the policymaking process.

A focus of the 2013 delegation was to meet with Senate offices to encourage
support for the Integrated University Program.  At the time of the Delegation,
the House Energy and Water appropriations bill had passed, with funding for IUP
included.  The Senate bill, which did not include IUP funding, had left
committee but not yet reached the floor for a vote.

Each NESD is a new and unique delegation. For the first time, the 2013
Delegation had meetings the Department of State and four NRC Commissioners,
including Chairman Macfarlane.  In addition, in keeping with at least the past
two delegations, the 2013 NESD met with NEI senior staff, executive branch
officials from DOE (including Assistant Secretary for Nuclear Energy Pete
Lyons), Office of Management and Budget (OMB), NRC, Annie Caputo- a Professional
Staff member at House Energy and Commerce Committee and the United States House
of Representatives, and the current ANS fellow Vincint Esposito.  The main
efforts of the Delegates were spent meeting with Members of Congress and their
staffers; the NESD was successful in meeting with or submitting policy
statements to every Senate office and a sizable number of House offices.

The 2013 NESD was successful in its twin purposes – to educate and inspire a
group of talented, young nuclear engineers about the policy-making world of
Washington, DC; and to convey the thoughts, opinions, and interests of nuclear
engineering students to policymakers.  It was an experience none of the
delegates will soon forget, and we are extremely grateful for the support we
received from NEI, ANS, and everyone who took time to meet with us.

%%%%%% schedule and events %%%%%%%%%%
\section{2013 Schedule}
\subsubsection*{Saturday, July 6th}
\begin{itemize}
  \item 7:00pm - 9:00pm - Orientation Dinner, The Dubliner
\end{itemize}

\subsubsection*{Sunday, July 7th}
\begin{itemize}
  \item 8:00AM - 12:00PM Policy Statement Writing, Hotel George
  \item 12:00PM - 1:30PM Lunch, Union Pub
  \item 1:30PM - 5:00PM Policy Statement Writing, Hotel George
\end{itemize}

\subsubsection*{Monday, July 8th}
\begin{itemize}
  \item 9:00am - 1:30pm - Meetings at AREVA
  \item 2:00pm - 5:00pm - Meetings at NEI
  \item 6:30pm - 9:00pm - Dinner with Lara Pierpoint, Ron Faibish, at Cafe Berlin
\end{itemize}

\subsubsection*{Tuesday, July 9th}
\begin{itemize}
  \item 9:00am - 11:30am - Meetings at the DOE
  \item 12:00pm - 1:30pm - Lunch with Pete Lyons
  \item 2:00pm - 6:00pm - Meetings with 3 NRC Commissioners
  \item 7:00pm - 9:00pm - Student Mixer
\end{itemize}

\subsubsection*{Wednesday, July 10th}
\begin{itemize}
  \item 9:00am - 12:00pm - Meetings at the State Department with Gilbert Brown
  \item 12:00pm - 3:00pm - Lunch at National Academy of Sciences
  \item 4:00pm - 5:00pm - Meetings at OMB with Christine MacDonald
  \item 6:00pm - 8:30pm - Dinner with NRC Commissioner Bill Magwood
\end{itemize}

\subsubsection*{Thursday, July 11th}
\begin{itemize}
  \item 8:00am - 5:00pm - Congressional Hill Visits
  \item 6:30pm - 8:30pm - Dinner with Annie Caputo and Vincint Esposito (ANS
    Fellow)
\end{itemize}

\subsubsection*{Friday, July 12th - Departure}
\begin{itemize}
  \item 8:00am - 5:00pm - Congressional Hill Visits
\end{itemize}

\section{Meetings \& Events}

\subsection*{Sunday, July 7th}

\subsubsection*{Policy Statement Writing}

The entire group gathered in a meeting room at the hotel and spent the first
hour or so talking about what they considered to be the important issues facing
nuclear engineering education. After everyone had a chance to express their
thoughts, the Chairs gave a quick recap of everything that had been
discussed. The delegates emphasized two subjects, continued funding of the
Integrated University Program (IUP) and passage of the Nuclear Waste
Administration Act (NWAA). A number of other issues were addressed, including
domestic fusion research funding, energy policy, nuclear export agreements, and
neutron detectors for port security. After the break for lunch, the delegates
divided back into groups and drafted each section. At the end of the day the
group edited the sections together. The delegates read the combined document
agreed upon its content by consensus.  

\subsection*{Monday, July 8th}

\subsubsection*{AREVA Meeting} 

On Monday morning, the delegates had their first meeting with the governmental
affairs staff of Areva. It began with a general presentation by the Vice
President of Governmental Affairs on Areva's international business
portfolio. The Areva staff gave a presentation about the MOX project and the
importance to our joint arms reduction commitments with Russia. They also
discussed the business importance of being active in policy discussions in
D.C. The morning concluded with a discussion with Mary Alice Hayward on her
experiences in nonproliferation and the importance of technical expertise in
international agreements.

\subsubsection*{NEI Meeting} 

On Monday afternoon, the delegation visited the Nuclear Energy Institute. Leslie
Barbour and other staff members described the function of NEI and how the
organization operates. They then discussed the importance of building and
maintaining relationships with congressional staff. They also explained NEI's
surveys and data on the nuclear workforce. We ended the meeting discussing our
policy statement and received helpful feedback.

\subsubsection*{Dinner with Congressional Fellows}

On Monday evening, the delegation had dinner with Lara Pierpoint (AAAS
Congressional Fellow working for Senator Ron Wyden, D-OR) and Ron Faibish
(Science Fellow for the Senate Committee on Energy and Natural Resources). These
young scholars shared their experiences on the Hill and specifically spoke about
their efforts in crafting the Nuclear Waste Administration Act.

\subsection*{Tuesday, July 9th}

\subsubsection*{DoE Meeting}

On Tuesday morning, the delegation visited the Department of Energy to meet with
the Office of Nuclear Energy (NE). The program leads for NE’s Fuel Cycle R\&D,
Light Water Reactor Technologies, and Nuclear Engineering University Program
(NEUP) discussed the myriad of funding and programmatic opportunities for
research provided by DOE NE. The staff discussed some of the priorities in
specific research areas and the importance of IUP funding. Brad Williams spoke
about the various forms of funding for graduate education, discussing the DOE
NEUP Scholarships and Fellowships (funded through the IUP) and the research
grants that provide graduate research assistantships at specific
universities. We then had lunch with Dr. Pete Lyons, Assistant Secretary of
Energy for Nuclear Energy, who shared his perspective on how D.C. operates and
the importance of nuclear engineering research to our nation's future.

\subsubsection*{NRC Meeting}

On Tuesday afternoon, the delegation went to the Nuclear Regulatory Commission
headquarters in Rockville, MD. The delegation met with NRC education staff, who
described the various methods by which the NRC provides educational enrichment
opportunities to universities - through graduate fellowships, undergraduate
scholarships, and curriculum development grants. The staff discussed internal
metrics used to ensure the effectiveness of the programs and reiterated that
none of these programs would be possible without the IUP. We also had the great
fortune of meeting with three of the commissioners on Tuesday afternoon
Commissioner George Apostolakis, Commissioner William C. Ostendorff, and
Chairman Allison M. Macfarlane. We had very robust discussions with all three
commissioners about many different topics, including Small Modular Reactors,
Gen-IV technology, Linear No-Threshold Dose, IUP funding, and commercial
reprocessing. The commissioners shared diverse perspectives from varied
backgrounds, but all are filtered through the lens of the NRC.

\subsection*{Wednesday, July 10th}

\subsubsection*{Department of State Meeting}

One Wednesday morning, the delegation met Dr. Gilbert Brown and Ryan Taugher at
the Department of State. Dr. Brown, a professor at University of Mass.-Lowell
and current Foster Fellow at the Department of State, explained the concept
``Team USA'' and the importance of the international framework on nuclear
security. Mr. Taugher explained the structure of the Department of State and the
framework for the Partnership for Nuclear Security (PNS). Through the PNS, the
Department of State establishes cooperative partnerships with other nations to
support the peaceful use of nuclear energy and achieve mutually beneficial
nuclear safety, security, and nonproliferation objectives.
 
\subsubsection*{OMB Meeting}

The Delegation’s annual meeting with the OMB on Wednesday afternoon was the most
notable in recent memory. Ms. Christine MacDonald, one of the employees
responsible for allocating DOE funds, discussed current budgeting. A
representative from the National Science Foundation (NSF) discussed the
President’s STEM initiative, NSF fellowships, NEUP, and IUP. The Delegation
provided recent statistical evidence showing that the majority of nuclear
engineering graduate students work for the U.S. Government after completing
their studies, effectively communicated how NEUP and IUP have helped increase
the attractiveness of the nuclear field to graduate students, and provided
numerous examples of how IUP has allowed students to conduct important nuclear
research that would not otherwise be funded.
 
\subsubsection*{Dinner with NRC Commissioner Magwood}

On Wednesday evening, the Delegation had dinner with Commissioner Magwood, who
cares deeply about student issues. We discussed our meetings with the other
three Commissioners, recent events in the nuclear industry (shutdown of San
Onofre and Kewaunee), and the IUP and the NRC’s role in education and workforce
development. Magwood imparted wisdom from his past experiences as chairman of
the Generation IV International Forum (GIF) and working with the Fast Flux Test
Facility (FFTF).

\subsection*{Thursday, July 11th and Friday, July 12th}

\subsubsection*{Hill Visits}

On Thursday and Friday, the Delegation accomplished its main objective on the
Hill. Delegates met with or dropped by the offices of all 100 Senators and 43
Representatives. The delegates also met with 5 senators (Maria Cantwell, D-WA;
Patty Murray, D-WA; Jeff Merkley, D-OR; Mike Lee, R-UT; Tom Harkin, D-IA) and 5
representatives (Ralph Hall, R-TX; Steve Stivers, R-OH; Chris Gibson, D-NY; John
Garamendi, D-CA; Bill Flores, R-TX) in person. The remainder of meetings took
place with congressional staffers and committees. Our presence was well received
and there were many conversations and an overall interest in learning more,
leaving us with the impression that our efforts indeed left a mark of influence
this year. In particular, as our visits earlier in the week to the NEI, DOE-ONE,
and NRC showed broad support for the NWAA, many of the Congressional offices
were interested in learning more about the bill or already supported it. Some
memorable discussions included one with Senator Feinstein’s office, which is
working on the NWAA. They stated that nuclear waste storage issues are more
political than technological, and they are open to implementing nuclear power in
the future to reduce carbon emissions if a waste storage solution is adopted. In
addition to discussion of our statement, delegates also had the opportunity to
discuss other nuclear-related topics and applications such as nuclear
desalination and nuclear development of U.S. oil shale. Delegates established
relationships that will be extremely valuable in the future, and we look forward
to observing the Delegation’s impact as the year progresses.

\subsubsection*{Thursday Dinner}

Thursday night we were pleased to have dinner with ANS/AAAS fellow Vincent
Esposito and Annie Caputo of the Senate Committee on Environment and Public
Works. Mr. Esposito gave some great career advice and talked about some of the
legislation he’s been working on.  

\subsubsection*{Friday Breakfast}

The following morning, we had breakfast with Leslie Barbour of NEI and Craig
Piercy, the ANS Washington Representative. Mr. Piercy talked about the ongoing
challenges with Yucca mountain and the need for people with technical experience
coming to Washington.


\newpage
\section{Feedback for Next Year}

Most of the feedback from the delegates regarding the 2013 NESD was positive.  A
sampling of positive statements from the final Delegation meeting is included
here:

\begin{itemize}
\item The organization of this years Delegation received high praise
\item Preparation for key meetings, such as OMB, was much improved over the
  previous years Delegation
\item Delegates all thought that the meetings set up for the Delegation were
  outstanding and invaluable
\end{itemize}

Some ideas that were suggested for consideration are listed here:
\begin{itemize}
\item Delegates should be encouraged to contact their university Washington
  liaisons for advice and potential coordination with existing efforts.
\item Attempts should be made to set up meetings with Appropriations Committee
  staff.  Several delegates enthusiastically endorsed this idea.
\item An intro/preparatory meeting with a budget expert would be useful,
  especially for delegates not already familiar with the intricacies of the
  federal budget process.
\item An idea of having the Delegation extend from mid-week to mid-week rather
  than beginning on a weekend was raised.  This would allow for some of the
  preparatory meetings to occur before the statement-writing session.  Some
  delegates felt that information received at meetings could have been
  fruitfully incorporated into the statement-writing process.
\item For next years Delegation, formal thank you and follow-up letters could be
  sent to key people with whom the delegation meets.
\end{itemize}

Other suggestions focused on preparing future delegates, such as additions and
improvements to the Guidebook prepared this year.  The Guidebook was well
received, and the delegates found it useful.

All in all, the feedback from the delegates was that the 2013 Nuclear
Engineering Student Delegation was a success.  The delegates’ suggestions for
improvement focused on ways that next years delegation can be made even more
effective.  All delegates left the week with greatly increased understanding of
the importance of being involved in the legislative process, as well as renewed
passion for doing whatever they can to help support nuclear education programs.

\newpage
\section{Conclusion}

The 2013 Nuclear Engineering Student Delegation was very successful in educating
policymakers on the needs for supporting nuclear engineering education and the
members of the delegation on the political process. During the delegations time
in Washington, D.C., members were able to speak for nuclear science and
engineering students throughout the country and show their enthusiasm for
nuclear professions. This years delegation was fortunate to meet with some very
influential members and advocates of the nuclear community. The NESD was also
very successful on Capitol Hill as they were able to visit or meet with the
offices of numerous members of the House of Representatives and most sitting
members of the Senate during their time in Washington, D.C.

The primary goal of this year’s delegation was to express to policymakers the
need for the Integrated University Program (IUP) that provides fellowships and
scholarships through DOE/NE and NRC to nuclear science and engineering
students. Currently, the IUP has been funded under the House's Water \& Energy
Appropriations bill. It is still uncertain whether or not the senate will
include the IUP in their appropriations bill.p

All members of the 2013 NESD were very pleased with their experience in
Washington, D.C. Delegates who will be in school next year look forward to the
possibility of being part of the 2014 NESD, which will be chaired by Nicholas
Thompson (Rensselar Polytechnic Institute). Finally, the members of the
delegation would like to thank the Nuclear Energy Institute, American Nuclear
Society, and all of the universities of the delegates for their help and
support.

\end{doublespace}
\end{document}
