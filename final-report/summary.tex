The 2013 Nuclear Engineering Student Delegation spent July 6-12 in Washington,
D.C. learning about the policy making process and advocating the issues related
to students in nuclear science and technology. The delegation this year included
sixteen students from twelve universities, and advocacy focused on continued
funding of the Integrated University Program and passage of the Nuclear Waste
Administration Act.

The first activity undertaken by the delegation was the drafting of a policy
statment, advocating for our constituency, to be delivered to policymakers and
their staffs later in the week. The delegation then met with a number of players
in the policy-making realm, including Areva and NEI, representatives of the
nuclear industry; DOE, NRC, and OMB, representatives of the executive branch;
and legislative members and their staff. The delegation was fortunate this year
to meet four out of the five commissioners of the NRC as well as Assistant
Secretary for Nuclear Energy, Pete Lyons.

The 2013 NESD was successful in its twin purposes – to educate and inspire a
group of talented, young nuclear engineers about the policy-making world of
Washington, DC, and to convey the thoughts, opinions, and interests of nuclear
engineering students to policymakers. The delegates expressed their keen
appreciation for such an opportunity, and the delegation is extremely grateful
for the support received from NEI, ANS, and their respective universities.
