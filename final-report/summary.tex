
The Nuclear Engineering Student Delegation marked another successful trip in
July 2013.  Delegates had unforgettable experiences learning about, and
participating in the policymaking process.

A focus of the 2013 delegation was to meet with Senate offices to encourage
support for the Integrated University Program.  At the time of the Delegation,
the House Energy and Water appropriations bill had passed, with funding for IUP
included.  The Senate bill, which did not include IUP funding, had left
committee but not yet reached the floor for a vote.

Each NESD is a new and unique delegation. For the first time, the 2013
Delegation had meetings the Department of State and four NRC Commissioners,
including Chairman Macfarlane.  In addition, in keeping with at least the past
two delegations, the 2013 NESD met with NEI senior staff, executive branch
officials from DOE (including Assistant Secretary for Nuclear Energy Pete
Lyons), Office of Management and Budget (OMB), NRC, Annie Caputo- a Professional
Staff member at House Energy and Commerce Committee and the United States House
of Representatives, and the current ANS fellow Vincint Esposito.  The main
efforts of the Delegates were spent meeting with Members of Congress and their
staffers; the NESD was successful in meeting with or submitting policy
statements to every Senate office and a sizable number of House offices.

The 2013 NESD was successful in its twin purposes – to educate and inspire a
group of talented, young nuclear engineers about the policy-making world of
Washington, DC; and to convey the thoughts, opinions, and interests of nuclear
engineering students to policymakers.  It was an experience none of the
delegates will soon forget, and we are extremely grateful for the support we
received from NEI, ANS, and everyone who took time to meet with us.
